\documentclass[11pt,letterpaper]{article}

\addtolength{\oddsidemargin}{-.875in}
\addtolength{\evensidemargin}{-.875in}
\addtolength{\textwidth}{1.75in}

\addtolength{\topmargin}{-.875in}
\addtolength{\textheight}{1.75in}

\usepackage[utf8]{inputenc}
\usepackage{caption} % for table captions
\usepackage{amsmath} % for multi-line equations and piecewises
\DeclareMathOperator{\sign}{sign}
\usepackage{graphicx}
\usepackage{relsize}
\usepackage{xspace}
\usepackage{verbatim} % for block comments
\usepackage{subcaption} % for subfigures
\usepackage{enumitem} % for a) b) c) lists
\newcommand{\Cyclus}{\textsc{Cyclus}\xspace}%
\newcommand{\Cycamore}{\textsc{Cycamore}\xspace}%
\newcommand{\deploy}{\texttt{d3ploy}\xspace}%
\newcommand{\Deploy}{\texttt{D3ploy}\xspace}%
\usepackage{tabularx}
\usepackage{color}
\usepackage{multirow}
\usepackage{float} 
\usepackage[acronym,toc]{glossaries}
\newacronym[longplural={metric tons of heavy metal}]{MTHM}{MTHM}{metric ton of heavy metal}
\newacronym{ANL}{ANL}{Argonne National Laboratory}
\newacronym{ARFC}{ARFC}{Advanced Reactors and Fuel Cycles}
\newacronym{ASME}{ASME}{American Society of Mechanical Engineers}
\newacronym{BOL}{BOL}{Beginning-of-Life}
\newacronym{CASL}{CASL}{Consortium for Advanced Simulation of Light Water Reactors}
\newacronym{CEA}{CEA}{Commissariat \`a l'\'Energie Atomique et aux \'Energies Alternatives}
\newacronym{DOE}{DOE}{Department of Energy}
\newacronym{EIA}{EIA}{U.S. Energy Information Administration}
\newacronym{EPA}{EPA}{Environmental Protection Agency}
\newacronym{GHG}{GHG}{greenhouse gas}
\newacronym{HTGR}{HTGR}{High Temperature Gas-Cooled Reactor}
\newacronym{IAEA}{IAEA}{International Atomic Energy Agency}
\newacronym{INL}{INL}{Idaho National Laboratory}
\newacronym{LANL}{LANL}{Los Alamos National Laboratory}
\newacronym{LWR}{LWR}{Light Water Reactor}
\newacronym{MCNP}{MCNP}{Monte Carlo N-Particle code}
\newacronym{MOOSE}{MOOSE}{Multiphysics Object-Oriented Simulation Environment}
\newacronym{MSRE}{MSRE}{Molten Salt Reactor Experiment}
\newacronym{MSR}{MSR}{Molten Salt Reactor}
\newacronym{NCSA}{NCSA}{National Center for Supercomputing Applications}
\newacronym{NEI}{NEI}{Nuclear Energy Institute}
\newacronym{NFC}{NFC}{Nuclear Fuel Cycle}
\newacronym{NGNP}{NGNP}{Next Generation Nuclear Plant}
\newacronym{NPRE}{NPRE}{Department of Nuclear, Plasma, and Radiological Engineering}
\newacronym{NRC}{NRC}{Nuclear Regulatory Commission}
\newacronym{NSSC}{NSSC}{Nuclear Science and Security Consortium}
\newacronym{ORNL}{ORNL}{Oak Ridge National Laboratory}
\newacronym{PI}{PI}{Principal Investigator}
\newacronym{syngas}{syngas}{synthetic gas}
\newacronym{TRISO}{TRISO}{Tristructural Isotropic}
\newacronym{UIUC}{UIUC}{University of Illinois at Urbana-Champaign}
\newacronym{US}{US}{United States}
%\newacronym{<++>}{<++>}{<++>}
%\newacronym{<++>}{<++>}{<++>}

\definecolor{bg}{rgb}{0.95,0.95,0.95}
\newcolumntype{b}{X}
\newcolumntype{f}{>{\hsize=.15\hsize}X}
\newcolumntype{s}{>{\hsize=.5\hsize}X}
\newcolumntype{m}{>{\hsize=.75\hsize}X}
\newcolumntype{r}{>{\hsize=1.1\hsize}X}
\usepackage{titling}
\usepackage[hang,flushmargin]{footmisc}
\renewcommand*\footnoterule{}
\usepackage{tikz}

\usetikzlibrary{shapes.geometric,arrows}
\tikzstyle{process} = [rectangle, rounded corners, 
minimum width=1cm, minimum height=1cm,text centered, draw=black, 
fill=blue!30]
\tikzstyle{arrow} = [thick,->,>=stealth]

\graphicspath{}
\title{Hydrogen Production Requirements}
%\author{Roberto E. Fairhurst Agosta}

\begin{document}
%	\begin{titlepage}
%		\maketitle
%		\thispagestyle{empty}
%	\end{titlepage}

\section{Central Hydrogen Production from Natural Gas without CO$_2$ Sequestration}

Desulfurized natural gas feedstock reacts with process steam over a nickel based catalyst inside of a system of high alloy steel tubes. The process is endothermic, and the metallurgy of the tubes limits the reaction temperature to 760 to 926 $^{\circ}$C.

\section{Hydrogen Production from Natural Gas}

Natural gas is the lowest cost alternative. Natural gas provides 95\% of the hydrogen used in refineries and chemical industry \cite{harstein_arthur_hydrogen_2003}. Syngas production. Syngas is a fuel gas mixture consisting of hydrogen, carbon monoxide, and some carbon dioxide.

Fischer-Tropsch synthesis is a catalyzed chemical reactor in which synthesis gas

\section{Exelon}

On going project that aims to provide a basis for converting baseload nuclear plants into hybrid plants that produce hydrogen.
Increases in variable wind and solar energy and low-cost natural gas impact baseload nuclear power generation stations make necessary a new operating paradigm for these plants.
Hydrogen production with nuclear energy may increase the plant revenue \cite{otgonbaatar_merchant_2019}.

\begin{figure}[] %H
	\centering
	\includegraphics[width=0.6\linewidth]{figures/exelon1.jpg}
	\hfill
	\caption{ \cite{otgonbaatar_merchant_2019}.}
	\label{fig:ghg}
\end{figure}

Fuelcell Energy.
NEL

\section{Hydrogen Storage}
There are three basic storage methods \cite{carriveau_hydrogen_2016}:
compressed hydrogen gas, liquid hydrogen, and solid storage of hydrogen. 
The first two consist of changing hydrogen's physical state in gaseous or liquid form.
The third method includes the following categories: physisorption in porous materials, absorbed on interstitial sites in a host metal, complex compounds, and metals and complexes with water.

The traditional hydrogen storage facilities are complicated, and liquid hydrogen requires a refrigeration unit, adding energy costs, and a resultant of 40\% loss in energy content \cite{carriveau_hydrogen_2016}.

\section{MTD Hydrogen Buses}

Fuel cell buses use a battery-dominant hybrid architecture where the batteries are large enough to handle all vehicle performance needs.
The fuel cell, Figure \ref{fig:nf2} acts like a continuous battery charger to extend the range of the vehicle.
The Xcelsior CHARGE H2 can travel up to 350 miles on a single refueling \cite{new_flyer_xcelsior_2019}.

\begin{figure}[] %H
	\centering
	\includegraphics[width=0.6\linewidth]{figures/newflyer1.jpg}
	\hfill
	\caption{ \cite{new_flyer_xcelsior_2019}.}
	\label{fig:nf1}
\end{figure}

\begin{figure}[] %H
	\centering
	\includegraphics[width=0.6\linewidth]{figures/newflyer2.jpg}
	\hfill
	\caption{ \cite{new_flyer_xcelsior_2019}.}
	\label{fig:nf2}
\end{figure}

\pagebreak
\bibliographystyle{plain}
\bibliography{bibliography}

\end{document}
