\documentclass[11pt,letterpaper]{article}

\addtolength{\oddsidemargin}{-.875in}
\addtolength{\evensidemargin}{-.875in}
\addtolength{\textwidth}{1.75in}

\addtolength{\topmargin}{-.875in}
\addtolength{\textheight}{1.75in}

\usepackage[utf8]{inputenc}
\usepackage{caption} % for table captions
\usepackage{amsmath} % for multi-line equations and piecewises
\DeclareMathOperator{\sign}{sign}
\usepackage{graphicx}
\usepackage{relsize}
\usepackage{xspace}
\usepackage{verbatim} % for block comments
\usepackage{subcaption} % for subfigures
\usepackage{enumitem} % for a) b) c) lists
\usepackage{tabularx}
\usepackage{color}
\usepackage{multirow}
\usepackage{float} 
\usepackage[acronym,toc]{glossaries}
\newacronym[longplural={metric tons of heavy metal}]{MTHM}{MTHM}{metric ton of heavy metal}
\newacronym{ANL}{ANL}{Argonne National Laboratory}
\newacronym{ARFC}{ARFC}{Advanced Reactors and Fuel Cycles}
\newacronym{ASME}{ASME}{American Society of Mechanical Engineers}
\newacronym{BOL}{BOL}{Beginning-of-Life}
\newacronym{CASL}{CASL}{Consortium for Advanced Simulation of Light Water Reactors}
\newacronym{CEA}{CEA}{Commissariat \`a l'\'Energie Atomique et aux \'Energies Alternatives}
\newacronym{DOE}{DOE}{Department of Energy}
\newacronym{EIA}{EIA}{U.S. Energy Information Administration}
\newacronym{EPA}{EPA}{Environmental Protection Agency}
\newacronym{GHG}{GHG}{greenhouse gas}
\newacronym{HTGR}{HTGR}{High Temperature Gas-Cooled Reactor}
\newacronym{IAEA}{IAEA}{International Atomic Energy Agency}
\newacronym{INL}{INL}{Idaho National Laboratory}
\newacronym{LANL}{LANL}{Los Alamos National Laboratory}
\newacronym{LWR}{LWR}{Light Water Reactor}
\newacronym{MCNP}{MCNP}{Monte Carlo N-Particle code}
\newacronym{MOOSE}{MOOSE}{Multiphysics Object-Oriented Simulation Environment}
\newacronym{MSRE}{MSRE}{Molten Salt Reactor Experiment}
\newacronym{MSR}{MSR}{Molten Salt Reactor}
\newacronym{NCSA}{NCSA}{National Center for Supercomputing Applications}
\newacronym{NEI}{NEI}{Nuclear Energy Institute}
\newacronym{NFC}{NFC}{Nuclear Fuel Cycle}
\newacronym{NGNP}{NGNP}{Next Generation Nuclear Plant}
\newacronym{NPRE}{NPRE}{Department of Nuclear, Plasma, and Radiological Engineering}
\newacronym{NRC}{NRC}{Nuclear Regulatory Commission}
\newacronym{NSSC}{NSSC}{Nuclear Science and Security Consortium}
\newacronym{ORNL}{ORNL}{Oak Ridge National Laboratory}
\newacronym{PI}{PI}{Principal Investigator}
\newacronym{syngas}{syngas}{synthetic gas}
\newacronym{TRISO}{TRISO}{Tristructural Isotropic}
\newacronym{UIUC}{UIUC}{University of Illinois at Urbana-Champaign}
\newacronym{US}{US}{United States}
%\newacronym{<++>}{<++>}{<++>}
%\newacronym{<++>}{<++>}{<++>}

\definecolor{bg}{rgb}{0.95,0.95,0.95}
\newcolumntype{b}{X}
\newcolumntype{f}{>{\hsize=.15\hsize}X}
\newcolumntype{s}{>{\hsize=.5\hsize}X}
\newcolumntype{m}{>{\hsize=.75\hsize}X}
\newcolumntype{r}{>{\hsize=1.1\hsize}X}
\usepackage{titling}
\usepackage[hang,flushmargin]{footmisc}
\renewcommand*\footnoterule{}
\usepackage{tikz}

\usetikzlibrary{shapes.geometric,arrows}
\tikzstyle{process} = [rectangle, rounded corners, 
minimum width=1cm, minimum height=1cm,text centered, draw=black, 
fill=blue!30]
\tikzstyle{arrow} = [thick,->,>=stealth]

\graphicspath{}
\title{Hydrogen Production}
%\author{Roberto E. Fairhurst Agosta}

\begin{document}
%	\begin{titlepage}
%		\maketitle
%		\thispagestyle{empty}
%	\end{titlepage}

\section{Electrolysis}

Electrolysis uses an electric current to split water into hydrogen and oxygen as shown in Figure \ref{fig:electro}.
The reaction takes place in a unit called an electrolyzer.
Electrolyzers consist of an anode and a cathode separated by an electrolyte.
A few classes of electrolyzer technologies, distinguished by their materials and functionality, include polymer electrolyte membrane, alkaline, and solid oxide electrolyzers \cite{doe_office_of_energy_efficiency_and_renewable_energy_hydrogen_2020}.

\begin{figure}[]
	\centering
	\includegraphics[width=0.55\linewidth]{figures/electrolysis.png}
	\hfill
	\caption{Production of hydrogen by electrolysis \cite{doe_office_of_energy_efficiency_and_renewable_energy_hydrogen_2020}.}
	\label{fig:electro}
\end{figure}

Solid oxide electrolyzers must operate at temperatures high enough for the solid oxide membranes to function properly (about 700$^{\circ}$-800$^{\circ}$C).
High temperature electrolysis (HTE) is more efficient.
The molar Gibbs energy of the reaction drops from 1.23eV (237 kJ/mol) at ambient temperature to 0.95 eV at 900$^{\circ}$C (183 kJ/mol) \cite{helmeth_high_2020}.
As a consequence, the process requires less electricity per kg of H$_2$ generated.

The use of heat at these elevated temperatures decreases the electricity needed.
Thermal energy rather than electricity converts water to steam and then the electricity dissociates the water at the cathode to form hydrogen molecules \cite{xu_introduction_2017}.

\begin{figure}[] %H
	\centering
	\includegraphics[width=0.6\linewidth]{figures/electrolysis2.png}
	\hfill
	\caption{Energy required for H$_2$O electrolysis \cite{helmeth_high_2020}.}
	\label{fig:nf2}
\end{figure}

\begin{figure}[] %H
	\centering
	\includegraphics[width=0.6\linewidth]{figures/electrolysis3.jpg}
	\hfill
	\caption{ Electrolysis \cite{doenitz_concepts_1982}.}
	\label{fig:electro3}
\end{figure}

3.37 kWh/m$^3$ H$_2$ (NTP = ambient temperature and pressure)
\cite{helmeth_high_2020}

Diffrent process modes i.e. endothermic, adiabatic or exothermic cell operation are possible and depend on the temperature of the feed steam available \cite{doenitz_concepts_1982}. Figure \ref{fig:electro3}

%https://www.sciencedirect.com/topics/engineering/high-temperature-electrolysis

\pagebreak
\bibliographystyle{plain}
\bibliography{bibliography}

\end{document}
