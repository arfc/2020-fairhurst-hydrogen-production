\documentclass{anstrans}
%%%%%%%%%%%%%%%%%%%%%%%%%%%%%%%%%%%
\title{Hydrogen Economy in Champaign-Urbana, IL}
\author{Roberto E. Fairhurst Agosta, Samuel G. Dotson, Kathryn D. Huff}

\institute{
University of Illinois at Urbana-Champaign, Dept. of Nuclear, Plasma, and Radiological Engineering\\
ref3@illinois.edu
}

%%%% packages and definitions (optional)
\usepackage{graphicx} % allows inclusion of graphics
\usepackage{booktabs} % nice rules (thick lines) for tables
\usepackage{microtype} % improves typography for PDF
\usepackage{xspace}
\usepackage{tabularx}
\usepackage{floatrow}
\usepackage{subcaption}
\usepackage{enumitem}
\usepackage{placeins}
\usepackage{amsmath}
\usepackage[acronym,toc]{glossaries}
\newacronym[longplural={metric tons of heavy metal}]{MTHM}{MTHM}{metric ton of heavy metal}
\newacronym{ANL}{ANL}{Argonne National Laboratory}
\newacronym{ARFC}{ARFC}{Advanced Reactors and Fuel Cycles}
\newacronym{ASME}{ASME}{American Society of Mechanical Engineers}
\newacronym{BOL}{BOL}{Beginning-of-Life}
\newacronym{CASL}{CASL}{Consortium for Advanced Simulation of Light Water Reactors}
\newacronym{CEA}{CEA}{Commissariat \`a l'\'Energie Atomique et aux \'Energies Alternatives}
\newacronym{DOE}{DOE}{Department of Energy}
\newacronym{EIA}{EIA}{U.S. Energy Information Administration}
\newacronym{EPA}{EPA}{Environmental Protection Agency}
\newacronym{GHG}{GHG}{greenhouse gas}
\newacronym{HTGR}{HTGR}{High Temperature Gas-Cooled Reactor}
\newacronym{IAEA}{IAEA}{International Atomic Energy Agency}
\newacronym{INL}{INL}{Idaho National Laboratory}
\newacronym{LANL}{LANL}{Los Alamos National Laboratory}
\newacronym{LWR}{LWR}{Light Water Reactor}
\newacronym{MCNP}{MCNP}{Monte Carlo N-Particle code}
\newacronym{MOOSE}{MOOSE}{Multiphysics Object-Oriented Simulation Environment}
\newacronym{MSRE}{MSRE}{Molten Salt Reactor Experiment}
\newacronym{MSR}{MSR}{Molten Salt Reactor}
\newacronym{NCSA}{NCSA}{National Center for Supercomputing Applications}
\newacronym{NEI}{NEI}{Nuclear Energy Institute}
\newacronym{NFC}{NFC}{Nuclear Fuel Cycle}
\newacronym{NGNP}{NGNP}{Next Generation Nuclear Plant}
\newacronym{NPRE}{NPRE}{Department of Nuclear, Plasma, and Radiological Engineering}
\newacronym{NRC}{NRC}{Nuclear Regulatory Commission}
\newacronym{NSSC}{NSSC}{Nuclear Science and Security Consortium}
\newacronym{ORNL}{ORNL}{Oak Ridge National Laboratory}
\newacronym{PI}{PI}{Principal Investigator}
\newacronym{syngas}{syngas}{synthetic gas}
\newacronym{TRISO}{TRISO}{Tristructural Isotropic}
\newacronym{UIUC}{UIUC}{University of Illinois at Urbana-Champaign}
\newacronym{US}{US}{United States}
%\newacronym{<++>}{<++>}{<++>}
%\newacronym{<++>}{<++>}{<++>}

\makeglossaries

\usepackage[printwatermark]{xwatermark}
\usepackage{xcolor}
\usepackage{graphicx}
\usepackage{lipsum}

\newcommand{\SN}{S$_N$}
\renewcommand{\vec}[1]{\bm{#1}} %vector is bold italic
\newcommand{\vd}{\bm{\cdot}} % slightly bold vector dot
\newcommand{\grad}{\vec{\nabla}} % gradient
\newcommand{\ud}{\mathop{}\!\mathrm{d}} % upright derivative symbol

\newcolumntype{c}{>{\hsize=.56\hsize}X}
\newcolumntype{b}{>{\hsize=.7\hsize}X}
\newcolumntype{s}{>{\hsize=.74\hsize}X}
\newcolumntype{f}{>{\hsize=.1\hsize}X}
\newcolumntype{a}{>{\hsize=.45\hsize}X}
%\usepackage[pagestyles]{titlesec}
%\titleformat*{\subsection}{\normalfont}
%\titleformat{\section}{\bfseries}{Item \thesection.\ }{0pt}{}

%\newwatermark[allpages,color=gray!50,angle=45,scale=3,xpos=0,ypos=0]{DRAFT}

\begin{document}
%%%%%%%%%%%%%%%%%%%%%%%%%%%%%%%%%%%%%%%%%%%%%%%%%%%%%%%%%%%%%%%%%%%%%%%%%%%%%%%%
\section{Introduction}

Climate change presents a threat that we should address swiftly. Decarbonizing the electric grid through variable renewable energy and nuclear power seems to be a remedy. Unfortunately, a carbon neutral electric grid will be insufficient to halt climate change because transportation contributes more to \gls{GHG} emissions than electricity. As seen in Figure \ref{fig:ghg}, transportation produced the most \glspl{GHG} in the US in 2017. Thus, decarbonizing transportation underpins global carbon reduction. Accordingly, the \gls{UIUC} has committed to the Illinois Climate Action Plan (iCAP) which aims to attain carbon neutrality by 2050 \cite{university_of_illinois_at_urbana-champaign_illlinois_2015}.

\begin{figure}[H]
	\centering
	\includegraphics[width=0.7\linewidth]{figures/total-ghg-2019-caption.jpg}
	\hfill
	\caption{Total U.S. GHG Emissions by Economic Sector in 2017 \cite{us_epa_sources_2020}.}
	\label{fig:ghg}
\end{figure}

One possible solution to reduce carbon emissions, and even achieve a net zero carbon production, is to develop a hydrogen economy as the state of California is currently doing \cite{brown_economic_2013}. 
Although using hydrogen does not produce CO$_2$, any hydrogen production method is only as carbon-free as the source of energy it relies on (electric, heat, or both).
Nuclear reactors present a clean energy option to produce H$_2$.

Micro-reactors are an innovative technology attractive for hydrogen production. Several micro-reactor designs are currently under development in the United States. These reactor concepts have three main features: they are factory fabricated, transportable, and self-regulating. All of the components are fully assembled in a factory and shipped out to the generation site, reducing capital costs and enabling rapid deployment. Simplified design concepts eliminate the need for many specialized operators and maintenance staff. Moreover, they utilize passive safety systems that prevent overheating or meltdown \cite{us-doe_ultimate_2019}.

The purpose of this abstract is to review and evaluate methods of hydrogen production for a hydrogen economy on a campus similar to the UIUC campus.
Section \ref{section:hydroprod} presents several methods and Section \ref{method} explains the methodology to calculate the mass of hydrogen required to fuel the Champaign-Urbana Mass Transit District (MTD) bus system and a portion of UIUC campus fleet service vehicles, as well as the mass of CO$_2$ produced by both fleets.

\section{Hydrogen Production Methods}
\label{section:hydroprod}

Some hydrogen production processes are: 
\begin{description}[font=$\bullet$\scshape\bfseries]
	\item[] Steam-Methane Reforming \cite{doe_office_of_energy_efficiency_and_renewable_energy_hydrogen_2020}
	\item[] Electrolysis \cite{doe_office_of_energy_efficiency_and_renewable_energy_hydrogen_2020}
	\item[] Iodine-Sulfur Thermochemical Cycle \cite{cea_gas-cooled_2006}
	\item[] Coal Gasification \cite{office_of_energy_efficiency_and_renewable_energy_coal_gas_2020}
	\item[] Thermochemical Water Splitting \cite{office_of_energy_efficiency_and_renewable_energy_thermo_water_2020}
\end{description}

The following subsections describe some of these methods.

\subsection{Steam Reforming}

Steam reforming (aka Natural Gas Reforming) is currently the least expensive way to produce hydrogen. This method separates hydrogen atoms from carbon atoms in methane (CH$_4$). This process results in carbon dioxide emissions.
Steam reforming is a mature production process that uses high-temperature steam (700$^{\circ}$C-1000$^{\circ}$C) to produce hydrogen from a methane source. Methane reacts with steam under 3-25 bar pressure in the presence of a catalyst to produce hydrogen, carbon monoxide, and a small portion of carbon dioxide. The reaction is endothermic and requires the supply of heat to convert methane and water to carbon monoxide and hydrogen gas with the following balance equation \cite{doe_office_of_energy_efficiency_and_renewable_energy_hydrogen_2020},

\begin{align}
CH_4 + H_2O + heat & \rightarrow CO + 3H_2 .
\label{eq:1}
\end{align}

A secondary reaction known as water-gas shift reaction occurs given by the balance equation,
\begin{align}
CO + H_2O & \rightarrow CO_2 + H_2
\label{eq:2}
\end{align}
producing CO$_2$ and more hydrogen.

\subsection{Electrolysis}

Electrolysis uses an electric current to split water into hydrogen and oxygen as shown in Figure \ref{fig:electro}. The reaction takes place in a unit called an electrolyzer. Electrolyzers consist of an anode and a cathode separated by an electrolyte. A few classes of electrolyzer technologies, distinguished by their materials and functionality, include polymer electrolyte membrane, alkaline, and solid oxide electrolyzers \cite{doe_office_of_energy_efficiency_and_renewable_energy_hydrogen_2020}.

\begin{figure}[H]
	\centering
	\includegraphics[width=0.55\linewidth]{figures/electrolysis.png}
	\hfill
	\caption{Production of hydrogen by electrolysis \cite{doe_office_of_energy_efficiency_and_renewable_energy_hydrogen_2020}.}
	\label{fig:electro}
\end{figure}

Solid oxide electrolyzers must operate at temperatures high enough for the solid oxide membranes to function properly (about 700$^{\circ}$-800$^{\circ}$C). The use of heat at these elevated temperatures decreases the electricity needed to produce hydrogen from water.
Thermal energy rather than electricity converts water to steam and then the electricity dissociates the water at the cathode to form hydrogen molecules \cite{xu_introduction_2017}.

\subsection{Iodine-Sulfur Thermochemical Cycle}

The most efficient methods operate at considerably high temperatures, typically above 900$^{\circ}$C. Sulfur-based cycles (Figure \ref{fig:isulfur}) use a sulfuric acid dissociation reaction that only works above 870$^{\circ}$C and whose efficiency increases with temperature \cite{cea_gas-cooled_2006}. The sulfur-iodine (SI) cycle results the best cycle for coupling to a high temperature reactor (HTR) due to its high efficiency. A General Atomics experiment has operated multiple times to produce hydrogen. The production was at a rate of 75 L/min. A scale-up of the process using a 50 MWth Nuclear Reactor could produce 12000 kg/day of hydrogen \cite{benjamin_russ_sulfur_2009}.
Another example of hydrogen production is by the Next Generation Nuclear Plant (NGNP) \cite{macdonald_ngnp_2003} which aims to produce 500 kg/h of H$_2$ by using 50 MWth \cite{cea_gas-cooled_2006}.

\begin{figure}[]
	\centering
	\includegraphics[width=0.9\linewidth]{figures/iodine-sulfur.png}
	\hfill
	\caption{Production of hydrogen by iodine-sulfur thermochemical cycle \cite{cea_gas-cooled_2006}.}
	\label{fig:isulfur}
\end{figure}

\section{Methodology}
\label{method}

A gasoline gallon equivalent (GGE) is the amount of fuel that can generate equivalent energy to a gallon of gasoline. One kilogram of hydrogen is equivalent to one gallon of gasoline \cite{doe_office_of_energy_efficiency_and_renewable_energy_hydrogen_2020}. Burning a gallon of gasoline produces 19.64 lbs of CO$_2$ \cite{us_energy_information_administration_how_2014}. 
Similarly, a diesel gallon equivalent (DGE) has the same amount of energy as a gallon of diesel. Approximately, a DGE is 113\% of a GGE \cite{alternative_fuels_data_center_fuel_2014}, then 1.13 kg of hydrogen is equivalent to one gallon of diesel.
A gallon of diesel produces 22.38 lbs of CO$_2$ \cite{us_energy_information_administration_how_2014}. 
Table \ref{tab:meth} summarizes this information.

\begin{table}[]
	\centering
    \caption{GGE, DGE, and CO$_2$ produced.}
    \label{tab:meth}
	\begin{tabular}{l|lll}
	\hline
	                 & Hydrogen & Gasoline    & Diesel      \\ \hline
	GGE              & 1 kg     & 1 gallon    & 0.88 gallon \\
	DGE              & 1.13 kg  & 1.13 gallon & 1 gallon    \\
    CO$_2$ produced  & -        & 19.64 lbs   & 22.38 lbs   \\ \hline

	\end{tabular}
\end{table}

\section{Results}

Figure \ref{fig:mtdfuel} shows the gallons of diesel purchased every day by MTD in a year. The data go from July 1st of 2018 to June 30th of 2019 \cite{mtd_irecords_2019}. The calculations assume that MTD consumed the purchased fuel on the same day.
Table \ref{tab:h2req} lists the mass of hydrogen required to supply the MTD fleet. Average gallons per day refers to the total amount of fuel consumed in a year averaged in 365 days. 

\begin{figure}[H]
	\centering
	\includegraphics[width=1.05\linewidth]{figures/mtd-fuel-consumption.png}
	\hfill
	\caption{Diesel gallons consumed each day by MTD from July 1, 2018 to June 30, 2019 \cite{mtd_irecords_2019}.}
	\label{fig:mtdfuel}
\end{figure}

The UIUC fleet includes both passenger and service vehicles \cite{uiuc_institute_for_sustainability_energy_and_environment_increase_2020}. The calculations consider only the portion of the fleet that operates in town and consumes gasoline \cite{uiuc_personnal_communication}. Figure \ref{fig:uiucfuel} presents the daily consumption of unleaded gasoline by the UIUC fleet in a year. The data go from January 1st of 2019 to December 31st of 2019. The fleet also uses diesel and ethanol but in smaller proportions that future analysis will take into account.
Table \ref{tab:h2req} lists the hydrogen required to supply the UIUC fleet.

\begin{figure}[H]
	\centering
	\includegraphics[width=0.95\linewidth]{figures/uiuc-unleaded.png}
	\hfill
	\caption{Gasoline gallons consumed each day by the UIUC fleet from January 1, 2019 to December 31, 2019 \cite{uiuc_personnal_communication}.}
	\label{fig:uiucfuel}
\end{figure}

\begin{table}[H]
	\centering
    \caption{Hydrogen required and CO$_2$ produced by MTD and UIUC fleets.}
    \label{tab:h2req}
\begin{tabular}{l|rr}
\hline
                   & MTD (Diesel)   & UIUC (Gasoline)  \\ \hline
Average gal/day    & 1,971.8        & 251.8            \\
kg of H$_2$/day    & 2,228.2        & 251.8            \\
CO$_2$ (lbs/day)   & 44,129.5       & 4,945.3          \\
gal/year           & 719,717.6      & 91,925.1         \\
kg of H$_2$/year   & 813,280.9      & 91,925.1         \\
CO$_2$ (lbs/year)  & 16,107,279.9   & 1,805,408.9      \\ \hline
\end{tabular}
\end{table}

Table \ref{tab:h2req} also shows the CO$_2$ emitted by MTD and UIUC fleets.
Both fleets combined would consume 2480 kg/day of H$_2$. A micro-reactor feeding an iodine-sulfur thermochemical cycle would need 10 MWth of power to meet that daily average demand of hydrogen.

\section{Conclusion}

MTD and UIUC fleets combined emit around 9 thousand tons of CO$_2$ per year in Champaign-Urbana. This has negative effects on the environment and intensifies climate change. The University of Illinois is leading by example and actively working to reduce GHG emissions on its campus. Switching to a hydrogen economy could be the answer to reducing CO$_2$ from transportation.

Nuclear energy could contribute as well. Some energy sources are not entirely emissions free. A 10 MWth micro-reactor would ease the CO$_2$ emissions on campus by generating energy for H$_2$ production regardless of weather conditions (in contrast with renewables). Additionally, the most efficient hydrogen production methods run at high temperatures, another reason nuclear is appealing.

\section{Acknowledgements}

Roberto E. Fairhurst Agosta and Prof. Huff are supported by the Nuclear Regulatory Commission (NRC) Faculty Development Program (award NRC-HQ-84-14-G-0054 Program B). Samuel G. Dotson is supported by the NRC Graduate Fellowship Program. Prof. Huff is also supported by the Blue Waters sustained-petascale computing project supported by the National Science Foundation (awards OCI-0725070 and ACI-1238993) and the state of Illinois, the DOE ARPA-E MEITNER Program (award DE-AR0000983), the DOE H2@Scale Program (award), and the International Institute for Carbon Neutral Energy Research (WPI-I2CNER), sponsored by the Japanese Ministry of Education, Culture, Sports, Science and Technology.
Additionally, the authors would like to thank Beth Brunk from MTD and Pete Varney from UIUC Facilities and Services for their contributions to the development of this document.

%%%%%%%%%%%%%%%%%%%%%%%%%%%%%%%%%%%%%%%%%%%%%%%%%%%%%%%%%%%%%%%%%%%%%%%%%%%%%%%%
\bibliographystyle{ans}
\bibliography{bibliography}
\end{document}
