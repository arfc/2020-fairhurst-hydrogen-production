\documentclass{anstrans}
%%%%%%%%%%%%%%%%%%%%%%%%%%%%%%%%%%%
\title{Hydrogen Economy in Champaign-Urbana}
\author{Roberto E. Fairhurst Agosta, Sammuel Dotson, Kathryn D. Huff}

\institute{
University of Illinois at Urbana-Champaign, Dept. of Nuclear, Plasma, and Radiological Engineering\\
ref3@illinois.edu
}

%%%% packages and definitions (optional)
\usepackage{graphicx} % allows inclusion of graphics
\usepackage{booktabs} % nice rules (thick lines) for tables
\usepackage{microtype} % improves typography for PDF
\usepackage{xspace}
\usepackage{tabularx}
\usepackage{floatrow}
\usepackage{subcaption}
\usepackage{enumitem}
\usepackage{placeins}
\usepackage[acronym,toc]{glossaries}
\newacronym[longplural={metric tons of heavy metal}]{MTHM}{MTHM}{metric ton of heavy metal}
\newacronym{ANL}{ANL}{Argonne National Laboratory}
\newacronym{ARFC}{ARFC}{Advanced Reactors and Fuel Cycles}
\newacronym{ASME}{ASME}{American Society of Mechanical Engineers}
\newacronym{BOL}{BOL}{Beginning-of-Life}
\newacronym{CASL}{CASL}{Consortium for Advanced Simulation of Light Water Reactors}
\newacronym{CEA}{CEA}{Commissariat \`a l'\'Energie Atomique et aux \'Energies Alternatives}
\newacronym{DOE}{DOE}{Department of Energy}
\newacronym{EIA}{EIA}{U.S. Energy Information Administration}
\newacronym{EPA}{EPA}{Environmental Protection Agency}
\newacronym{GHG}{GHG}{greenhouse gas}
\newacronym{HTGR}{HTGR}{High Temperature Gas-Cooled Reactor}
\newacronym{IAEA}{IAEA}{International Atomic Energy Agency}
\newacronym{INL}{INL}{Idaho National Laboratory}
\newacronym{LANL}{LANL}{Los Alamos National Laboratory}
\newacronym{LWR}{LWR}{Light Water Reactor}
\newacronym{MCNP}{MCNP}{Monte Carlo N-Particle code}
\newacronym{MOOSE}{MOOSE}{Multiphysics Object-Oriented Simulation Environment}
\newacronym{MSRE}{MSRE}{Molten Salt Reactor Experiment}
\newacronym{MSR}{MSR}{Molten Salt Reactor}
\newacronym{NCSA}{NCSA}{National Center for Supercomputing Applications}
\newacronym{NEI}{NEI}{Nuclear Energy Institute}
\newacronym{NFC}{NFC}{Nuclear Fuel Cycle}
\newacronym{NGNP}{NGNP}{Next Generation Nuclear Plant}
\newacronym{NPRE}{NPRE}{Department of Nuclear, Plasma, and Radiological Engineering}
\newacronym{NRC}{NRC}{Nuclear Regulatory Commission}
\newacronym{NSSC}{NSSC}{Nuclear Science and Security Consortium}
\newacronym{ORNL}{ORNL}{Oak Ridge National Laboratory}
\newacronym{PI}{PI}{Principal Investigator}
\newacronym{syngas}{syngas}{synthetic gas}
\newacronym{TRISO}{TRISO}{Tristructural Isotropic}
\newacronym{UIUC}{UIUC}{University of Illinois at Urbana-Champaign}
\newacronym{US}{US}{United States}
%\newacronym{<++>}{<++>}{<++>}
%\newacronym{<++>}{<++>}{<++>}

\makeglossaries

\newcommand{\SN}{S$_N$}
\renewcommand{\vec}[1]{\bm{#1}} %vector is bold italic
\newcommand{\vd}{\bm{\cdot}} % slightly bold vector dot
\newcommand{\grad}{\vec{\nabla}} % gradient
\newcommand{\ud}{\mathop{}\!\mathrm{d}} % upright derivative symbol

\newcolumntype{c}{>{\hsize=.56\hsize}X}
\newcolumntype{b}{>{\hsize=.7\hsize}X}
\newcolumntype{s}{>{\hsize=.74\hsize}X}
\newcolumntype{f}{>{\hsize=.1\hsize}X}
\newcolumntype{a}{>{\hsize=.45\hsize}X}
\usepackage{titlesec}
\titleformat*{\subsection}{\normalfont}

\begin{document}
%%%%%%%%%%%%%%%%%%%%%%%%%%%%%%%%%%%%%%%%%%%%%%%%%%%%%%%%%%%%%%%%%%%%%%%%%%%%%%%%
\section{Introduction}

The electric grid is important to decarbonize in order to reach out carbon goals and halt climate change
(mention Icap see \cite{holcomb_fueling_2015}).
Unfortunately, as we have seen with France that alone will not be enough. Around 40\% (find reference and more accurate numbers) of France's CO$_2$ is from transportation.
Micro-reactors present a solution to this problem by producing clean H$_2$ and heat for use in other industrial processes.

UIUC campus fleet service vehicles
Champaign-Urbana Mass Transit District (MTD) bus system


Next section lists a few technologies to produce hydrogen.

\section{Hydrogen Production Technologies}

Some hydrogen production processes are: 
\begin{description}[font=$\bullet$\scshape\bfseries]
	\item[] Steam-Methane reforming (aka Natural Gas Reforming).
	\item[] Coal Gasification with CCS (aka Carbon Sequestration).
	\item[] High-Temperature Electrolysis.
	\item[] Electrolysis.
	\item[] .
\end{description}

The two most common methods for producing hydrogen are steam reforming and electrolysis.
Steam reforming is currently the least expensive way to produce hydrogen. This method separates hydrogen atoms from carbon atoms in methane (CH4). This process results in carbon dioxide emissions.
Electrolysis splits hydrogen from water using an electric current. The emissions are hydrogen and oxygen \cite{noauthor_production_2019}. 

\section{Steam Reforming}

Steam reforming is a mature production process that uses high-temperature steam (700$^{\circ}$C-1000$^{\circ}$C) to produce hydrogen from a methane source. Methane reacts with steam under 3-25 bar pressure in the presence of a catalyst to produce hydrogen, carbon monoxide, and a small portion of carbon dioxide. The reaction is endothermic and requires the supply of heat to occure \cite{noauthor_hydrogen_nodate}.
\begin{equation}
CH_4 + H_2O + heat \rightarrow CO + 3H_2
\end{equation}
A secondary reaction know as water-gas shift reaction occurs producing $CO_2$ and more hydrogen:
\begin{equation}
CO + H_2O \rightarrow CO_2 + H_2
\end{equation}

Even with the upstream process of producing hydrogen from natural gas as well as delivering and storing it for use in fuel cell electric vehicles, the process reduces the greenhouse emissions in half and the use of petroleum over 90\% in today's gasoline vehicles.

\section{Electrolysis}

Electrolysis is the process of using electricity to split water into hydrogen and oxygen, Fig. \ref{fig:electro}. The reaction takes place in a unit called electrolyzer. Electrolyzers consist of an anode and a cathode separated by an electrolyte. Different electrolyzers function in slightly different ways. A few types are polymer electrolyte membrane, alkaline, and solid oxide electrolyzers.

\begin{figure}[H]
	\centering
	\includegraphics[width=0.4\linewidth]{figures/electrolysis.png}
	\hfill
	\caption{Production of hydrogen by electrolysis.}
	\label{fig:electro}
\end{figure}

Solid oxide electrolyzers must operate at thermperatures high enough for the solid oxide membranes to function properly (about 700$^{\circ}$-800$^{\circ}$C). The use of heat at these elevated temperatures decreases the amount of electrical energy needed to produce hydrogen from water.


\section{Methodology}
A gasoline gallon equivalent (gge) is the amount of fuel that has the same amount of energy as a gallon of gasoline. One kilogram of hydrogen is equivalent to one gallon of gasoline \cite{noauthor_hydrogen_nodate}.
Similarly, a diesel gallon equivalent (dge) has the same amount of energy as a gallon of diesel. Approximately, a dge is 113\% of a gge \cite{noauthor_fuel_2014}, then 1.13 Kg of hydrogen is equivalent to one gallon of diesel.

\section{Results}

\section{Conclusion}

%%%%%%%%%%%%%%%%%%%%%%%%%%%%%%%%%%%%%%%%%%%%%%%%%%%%%%%%%%%%%%%%%%%%%%%%%%%%%%%%
\bibliographystyle{ans}
\bibliography{bibliography}
\end{document}

