\documentclass{anstrans}
%%%%%%%%%%%%%%%%%%%%%%%%%%%%%%%%%%%
\title{Hydrogen Economy in Champaign-Urbana, IL}
\author{Roberto E. Fairhurst Agosta, Samuel G. Dotson, Kathryn D. Huff}

\institute{
University of Illinois at Urbana-Champaign, Dept. of Nuclear, Plasma, and Radiological Engineering\\
ref3@illinois.edu
}

%%%% packages and definitions (optional)
\usepackage{graphicx} % allows inclusion of graphics
\usepackage{booktabs} % nice rules (thick lines) for tables
\usepackage{microtype} % improves typography for PDF
\usepackage{xspace}
\usepackage{tabularx}
\usepackage{floatrow}
\usepackage{subcaption}
\usepackage{enumitem}
\usepackage{placeins}
\usepackage[acronym,toc]{glossaries}
\newacronym[longplural={metric tons of heavy metal}]{MTHM}{MTHM}{metric ton of heavy metal}
\newacronym{ANL}{ANL}{Argonne National Laboratory}
\newacronym{ARFC}{ARFC}{Advanced Reactors and Fuel Cycles}
\newacronym{ASME}{ASME}{American Society of Mechanical Engineers}
\newacronym{BOL}{BOL}{Beginning-of-Life}
\newacronym{CASL}{CASL}{Consortium for Advanced Simulation of Light Water Reactors}
\newacronym{CEA}{CEA}{Commissariat \`a l'\'Energie Atomique et aux \'Energies Alternatives}
\newacronym{DOE}{DOE}{Department of Energy}
\newacronym{EIA}{EIA}{U.S. Energy Information Administration}
\newacronym{EPA}{EPA}{Environmental Protection Agency}
\newacronym{GHG}{GHG}{greenhouse gas}
\newacronym{HTGR}{HTGR}{High Temperature Gas-Cooled Reactor}
\newacronym{IAEA}{IAEA}{International Atomic Energy Agency}
\newacronym{INL}{INL}{Idaho National Laboratory}
\newacronym{LANL}{LANL}{Los Alamos National Laboratory}
\newacronym{LWR}{LWR}{Light Water Reactor}
\newacronym{MCNP}{MCNP}{Monte Carlo N-Particle code}
\newacronym{MOOSE}{MOOSE}{Multiphysics Object-Oriented Simulation Environment}
\newacronym{MSRE}{MSRE}{Molten Salt Reactor Experiment}
\newacronym{MSR}{MSR}{Molten Salt Reactor}
\newacronym{NCSA}{NCSA}{National Center for Supercomputing Applications}
\newacronym{NEI}{NEI}{Nuclear Energy Institute}
\newacronym{NFC}{NFC}{Nuclear Fuel Cycle}
\newacronym{NGNP}{NGNP}{Next Generation Nuclear Plant}
\newacronym{NPRE}{NPRE}{Department of Nuclear, Plasma, and Radiological Engineering}
\newacronym{NRC}{NRC}{Nuclear Regulatory Commission}
\newacronym{NSSC}{NSSC}{Nuclear Science and Security Consortium}
\newacronym{ORNL}{ORNL}{Oak Ridge National Laboratory}
\newacronym{PI}{PI}{Principal Investigator}
\newacronym{syngas}{syngas}{synthetic gas}
\newacronym{TRISO}{TRISO}{Tristructural Isotropic}
\newacronym{UIUC}{UIUC}{University of Illinois at Urbana-Champaign}
\newacronym{US}{US}{United States}
%\newacronym{<++>}{<++>}{<++>}
%\newacronym{<++>}{<++>}{<++>}

\makeglossaries

\newcommand{\SN}{S$_N$}
\renewcommand{\vec}[1]{\bm{#1}} %vector is bold italic
\newcommand{\vd}{\bm{\cdot}} % slightly bold vector dot
\newcommand{\grad}{\vec{\nabla}} % gradient
\newcommand{\ud}{\mathop{}\!\mathrm{d}} % upright derivative symbol

\newcolumntype{c}{>{\hsize=.56\hsize}X}
\newcolumntype{b}{>{\hsize=.7\hsize}X}
\newcolumntype{s}{>{\hsize=.74\hsize}X}
\newcolumntype{f}{>{\hsize=.1\hsize}X}
\newcolumntype{a}{>{\hsize=.45\hsize}X}
%\usepackage[pagestyles]{titlesec}
%\titleformat*{\subsection}{\normalfont}
%\titleformat{\section}{\bfseries}{Item \thesection.\ }{0pt}{}

\begin{document}
%%%%%%%%%%%%%%%%%%%%%%%%%%%%%%%%%%%%%%%%%%%%%%%%%%%%%%%%%%%%%%%%%%%%%%%%%%%%%%%%
\section{Introduction}

The Illinois Climate Action Plan (iCAP) focuses on making UIUC campus more sustainable. The iCAP goal is to attain carbon neutrality considerably sooner than 2050. As Figure \ref{fig:ghg} displays, transportation contributed the greatest amount of greenhouse emissions in the US in 2017. It is important to decarbonize transportation on UIUC campus in order to reach the iCAP goal and halt climate change \cite{noauthor_illlinois_2015}.

\begin{figure}[H]
	\centering
	\includegraphics[width=0.6\linewidth]{figures/total-ghg-2019-caption.jpg}
	\hfill
	\caption{Total U.S. Greenhouse Gas Emissions by Economic Sector in 2017 \cite{us_epa_sources_2020}.}
	\label{fig:ghg}
\end{figure}

One possible solution to reduce carbon emissions, and even achieve a net zero carbon production is to develop hydrogen economies as the state of California is currently doing \cite{brown_economic_2013}. This article presents a few hydrogen production methods. Some of those methods help reduce carbon emissions but do not eliminate the production of CO$_2$. Nuclear reactors introduce a solution to this problem by producing clean H$_2$ and heat for use in other industrial processes.

Micro-Reactors are an innovative technology that is very attractive for hydrogen production. This type of reactor has three main features: factory fabricated, transportable, and self-regulating. All the components fully assembled in a factory and shipped out to location will reduce capital costs and enable rapid deployment. Simple design concepts eliminate the need of a large number of specialized operators. Moreover, they will utilize passive safety systems that prevent overheating or meltdown \cite{noauthor_ultimate_2019}.

The purpose of this abstract is to review and evaluate methods of hydrogen production for a hydrogen economy at UIUC campus.
Section \ref{section:hydroprod} presents such methods and Section \ref{method} explains the methodology to calculate the amount of hydrogen required to fuel Champaign-Urbana Mass Transit District (MTD) bus system and UIUC campus fleet service vehicles, as well as the amount of CO$_2$ produced by both fleets.

\section{Hydrogen Production Methods}
\label{section:hydroprod}

Some hydrogen production processes are: 
\begin{description}[font=$\bullet$\scshape\bfseries]
	\item[] Steam-Methane Reforming (aka Natural Gas Reforming).
	\item[] Electrolysis.
	\item[] High-Temperature Electrolysis.
	\item[] Iodine-Sulfur Thermochemical Cycle.
	\item[] Coal Gasification with CCS (aka Carbon Sequestration).
	\item[] Solar Thermochemical Hydrogen Production.
\end{description}

The following subsections describe some of these methods.

\subsection{Steam Reforming}

Steam reforming is currently the least expensive way to produce hydrogen. This method separates hydrogen atoms from carbon atoms in methane (CH4). This process results in carbon dioxide emissions.
Steam reforming is a mature production process that uses high-temperature steam (700$^{\circ}$C-1000$^{\circ}$C) to produce hydrogen from a methane source. Methane reacts with steam under 3-25 bar pressure in the presence of a catalyst to produce hydrogen, carbon monoxide, and a small portion of carbon dioxide. The reaction is endothermic and requires the supply of heat to occur \cite{noauthor_hydrogen_nodate}:
\begin{equation}
CH_4 + H_2O + heat \rightarrow CO + 3H_2
\end{equation}
A secondary reaction known as water-gas shift reaction occurs producing $CO_2$ and more hydrogen:
\begin{equation}
CO + H_2O \rightarrow CO_2 + H_2
\end{equation}

Even with the upstream process of producing hydrogen from natural gas, this process reduces the greenhouse emissions in half \cite{noauthor_hydrogen_nodate}.

\subsection{Electrolysis}

Electrolysis is the process of using an electric current to split water into hydrogen and oxygen, Fig. \ref{fig:electro}. The reaction takes place in a unit called electrolyzer. Electrolyzers consist of an anode and a cathode separated by an electrolyte. Different electrolyzers function in slightly different ways. A few types are polymer electrolyte membrane, alkaline, and solid oxide electrolyzers.

\begin{figure}[]
	\centering
	\includegraphics[width=0.55\linewidth]{figures/electrolysis.png}
	\hfill
	\caption{Production of hydrogen by electrolysis.}
	\label{fig:electro}
\end{figure}

\subsection{High-Temperature Electrolysis (HTE)}

Solid oxide electrolyzers must operate at temperatures high enough for the solid oxide membranes to function properly (about 700$^{\circ}$-800$^{\circ}$C). The use of heat at these elevated temperatures decreases the amount of electrical energy needed to produce hydrogen from water. In the HTE process, nuclear thermal energy rather than electricity converts water to steam and then the electricity dissociates the water at the cathode to form hydrogen molecules \cite{xu_introduction_2017}.

\subsection{Iodine-Sulfur Thermochemical Cycle}

The concept of using nuclear heat and water allows the possibility of a sustainable production without greenhouse gases. The most simple and promising methods, in terms of efficiency, operate at very high temperatures, typically above 900$^{\circ}$C. For example, sulfur-based cycles (Fig. \ref{fig:isulfur}) use a sulfuric acid dissociation reaction that only works above 870$^{\circ}$C and whose efficiency increases with temperature \cite{cea_gas-cooled_2006}. The sulfur-iodine (SI) cycle results the best cycle for coupling to a high temperature reactor (HTR) due to its high efficiency. A General Atomics experiment has operated multiple times to produce hydrogen. The production was at a rate of 75 L/min. The same report estimates that a scale-up of the process using a 50 MWt Nuclear Reactor could produce 12000 kg/day of Hydrogen \cite{benjamin_russ_sulfur_2009}.
Another example of hydrogen production is by the Next Generation Nuclear Plant (NGNP) \cite{macdonald_ngnp_2003} which aims to produce 500 kg/h of H$_2$ by using 50 MWt \cite{cea_gas-cooled_2006}.

\begin{figure}[H]
	\centering
	\includegraphics[width=0.85\linewidth]{figures/iodine-sulfur.png}
	\hfill
	\caption{Production of hydrogen by iodine-sulfur themochemical cycle.}
	\label{fig:isulfur}
\end{figure}

\section{Methodology}
\label{method}

A gasoline gallon equivalent (GGE) is the amount of fuel that has the same amount of energy as a gallon of gasoline. One kilogram of hydrogen is equivalent to one gallon of gasoline \cite{noauthor_hydrogen_nodate}. Burning a gallon of gasoline produces 19.64 lbs of CO$_2$ \cite{noauthor_how_2014}. 
Similarly, a diesel gallon equivalent (DGE) has the same amount of energy as a gallon of diesel. Approximately, a DGE is 113\% of a GGE \cite{noauthor_fuel_2014}, then 1.13 Kg of hydrogen is equivalent to one gallon of diesel.
A gallon of diesel produces 22.38 lbs of CO$_2$ \cite{noauthor_how_2014}. 
Table \ref{tab:meth} summarizes this information.

\begin{table}[!h]
	\centering
    \caption{GGE, DGE, and CO$_2$ produced.}
    \label{tab:meth}
	\begin{tabular}{l|lll}
	\hline
	                 & Hydrogen & Gasoline    & Diesel      \\ \hline
	GGE              & 1 Kg     & 1 Gallon    & 0.88 Gallon \\
	DGE              & 1.13 Kg  & 1.13 Gallon & 1 Gallon    \\
    CO$_2$ produced  & -        & 19.64 lbs   & 22.38 lbs   \\ \hline

	\end{tabular}
\end{table}

\section{Results}

Figure \ref{fig:mtdfuel} shows the amount of diesel purchased every day by MTD in a year \cite{mtd_irecords_2019}. The calculations take into account the assumption that MTD consumed the purchased fuel on the same day.
Table \ref{tab:h2req} lists the required amounts of hydrogen to supply the MTD fleet. Average Gallons per day refers to the total amount of fuel consumed in a year averaged in 365 days. 

\begin{figure}[!h]
	\centering
	\includegraphics[width=1.05\linewidth]{figures/fuelconsumption.png}
	\hfill
	\caption{Quantity of gallons of diesel consumed each day by MTD from July 1st 2018 to June 30th 2019.}
	\label{fig:mtdfuel}
\end{figure}

For the case of the UIUC fleet, 227 passenger vehicles and 275 service vehicles compose the totality of the fleet \cite{noauthor_increase_2020}. The calculations consider only 108 vehicles chosen based on their annual mileage that consume 269 gasoline gallons per day \cite{holcomb_fueling_2015}. Table \ref{tab:h2req} lists the required amounts of hydrogen to supply the UIUC fleet. The table also shows the amount of CO$_2$ emitted by both fleets.

\begin{table}[]
	\centering
    \caption{Hydrogen required and CO$_2$ produced by MTD and UIUC fleets.}
    \label{tab:h2req}
\begin{tabular}{l|rr}
\hline
                          & MTD(Diesel)      & UIUC(Gasoline)   \\ \hline
Average gal/day           & 1,971.8          & 269.0            \\
Kg of H$_2$/day           & 2,228.2          & 269.0            \\
CO$_2$ emitted (lbs/day)  & 44,129.5         & 5,283.2          \\
Gal/year                  & 719,717.6        & 98,185.0         \\
Kg of H$_2$/year          & 813,280.9        & 98,185.0         \\
CO$_2$ emitted (lbs/year) & 16,107,279.9     & 1,928,353.4      \\ \hline
\end{tabular}
\end{table}

\section{Conclusion}

Transportation produces large amounts of CO$_2$ in Champaign-Urbana, Illinois, and the United States. This has negative effects on the environment, and intensifies climate change. The University of Illinois is leading by example and actively working to stop this on its campus. Switching to a hydrogen economy could be the answer to reducing CO$_2$ by transportation. Nuclear energy could contribute to achieve this goal as well. As seen before, some methods of producing hydrogen are not entirely emissions free. A micro-reactor on campus would ease the greenhouse gases emissions by producing hydrogen regardless the weather conditions (contrary to renewables).

Another clear advantage of producing the hydrogen with micro-reactors is decentralization. Hydrogen technologies still present many challenges such as transportation and storage \cite{office_of_energy_efficiency_and_renewable_energy_hydrogen_2020}\cite{office_of_energy_efficiency_and_renewable_energy_hydrogen_delivery_2020}. Consequently, producing fuel locally is a clear advantage, as transportation and storage in large scales are not a concern anymore.

%%%%%%%%%%%%%%%%%%%%%%%%%%%%%%%%%%%%%%%%%%%%%%%%%%%%%%%%%%%%%%%%%%%%%%%%%%%%%%%%
\bibliographystyle{ans}
\bibliography{bibliography}
\end{document}
